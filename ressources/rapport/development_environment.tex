\chapter{Development Environment and Conventions}
In this part, we will see the development environment we used, and the programming conventions we've applied.

\section{Development Environment}

\subsection{github}

To write this report and to be able to implement both models most effectively possible, we used a tool of hosting and management of program which is called Github. It is a tools of versionning which allows the collaborative work and allows a simplification of the methods of work. \\
We used it throughout the project to put the summaries of articles that we read, the tasks which we had to make, this report or still the implementation.\\
Our Git repository is located at the following address :
https://github.com/jetcheve/M2\_PER\_MMUAVGRA.\\\\

Our deposit decomposes into three branches :
\begin{itemize}
\item A master branch which is general to the whole deposit.
\item A branch called Develop. It is in this branch that there will be all the code for implementing different models. It also contains a test folder. 
\item An branch report, which as its name suggests, contain all the parts of our report and the pictures have inside it.
\end{itemize}

\subsection{Eclipse}


\section{Programming Conventions}
First, we've choosed to used the Java Coding Conventions, which we can see in the following link - http://www.oracle.com/technetwork/java/codeconventions-150003.pdf\\

Then, in order to create our documentation, we've used the Doxygen Convention, viewable here - \url{http://www.stack.nl/~dimitri/doxygen/manual/docblocks.html}\\
We've relied heavily on this documentation and have mostly employed these protocoles below.\\

\lstset{language=Java,basicstyle=\footnotesize,commentstyle=\color{blue}}
\begin{lstlisting}[frame=trBL, title=Doxygen Convention for classes]
/**
* @class name of the class
* @brief Description of herself
*/
\end{lstlisting}

\lstset{language=Java,basicstyle=\footnotesize,commentstyle=\color{blue}}
\begin{lstlisting}[frame=trBL, title=Doxygen Convention for methods]
/**
* @brief Description of the method
* @param the parameters and their descriptions
* @return the description of what return the method (optional)
*/
\end{lstlisting}

\lstset{language=Java,basicstyle=\footnotesize,keywordstyle=\color{red},commentstyle=\color{blue}}
\begin{lstlisting}[frame=trBL, title=Doxygen Convention for members]
int var; /**< Detailed description after the member */
\end{lstlisting}
~\\
We've also resorted to programming conventions that we defined between us.\\
For example, when a part of a code, was not finished yet, we put the following lines above the concerning part.\\

\lstset{language=Java,basicstyle=\footnotesize,commentstyle=\color{blue}}
\begin{lstlisting}[frame=trBL, title=Programming convention for unfinished code]
/**
* @TO_DO
* Description
*/
\end{lstlisting}
~\\
We've also used a convention for the bugs found and wrote these protocoles, depending on whether the bug was resolved or not.\\
We used it, in line with the Bug Tracking of GitHub.\\

\lstset{language=Java,basicstyle=\footnotesize,commentstyle=\color{blue}}
\begin{lstlisting}[frame=trBL, title=Programming convention for bugs]
/**
* @BUG
* @Unfinished/finished
* Description
*/
\end{lstlisting}
~\\
To finish, we've created the five essential files for a project : \\
\begin{itemize}
\item INSTALL.txt  : Installation instructions for the project,
\item LICENCE.txt  : Licence and copyright \copyright{} of the project,
\item README.txt   : General description of the project,
\item AUTHORS.txt  : Authors of the project,
\item MANIFEST.txt : Tree structure and files list of the project.
\end{itemize}

