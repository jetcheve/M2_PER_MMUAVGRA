\part{Assessment}

During this project of studies and researches, we studied an article which is called \textit{Mobility Models for UAV Group Reconnaissance Applications}. After reading it, we have identified the main issue of the article. Then, we did a study of existing models described in the article, here the mobility models. This article describes two principal models : a model based on random movements of UAV and a model based on the use of pheromones.\\\\
We implemented both models mentioned above, more another model (random Waypoint) that our clients have asked us. We compared these three models to scan an area. We concluded that in the beginning random models are much more effective, but after some time, they remain stable. Pheromone model is more effective than the others models to reach 100\% of scan. The pheromone model is a good model for scan coverage and reconnaissance scenario. But experiment conditions are unrealistic.\\\\
Our clients asked us to compare the scenarios in this article with other mobility models of the others groups.
Unfortunately, we were not able to do this comparison because different models are not used in the same types of scenarios. Semi-Random-Circular-Movement and Distributed Pheromone Repel do the scan coverage whereas Smooth turn is used for airborne Networks.


