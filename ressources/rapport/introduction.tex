\chapter*{Introduction}

As part of our Master Computer formation at the University of Bordeaux 1 particularly in the matter Study and Research Project, we had to study a research article in groups of 5 people. Every group had to choose a subject of research among those proposed. Our group chose a topic related to the drones. The title of this article is \textit{Mobility Models for UAV Group Reconnaissance Applications} written in collaboration by \textit{Erik Kuiper} and \textit{Simin Nadjm-Tehrani} published in 2006. The customers who suggested us studying this article are misters Serge Chaumette and Vincent Autefage. Our teaching assistant is mister Pascal Desbarats.
\\\\

This project, for a period of 3 months, we were able to see the progress of the study of a research article, its reading in English, with the implementation of an algorithm, and going through a study of the existing.
The aim of this matter is to make us study a research article, to extract an algorithm from it and to implement it.
We must therefore study a model that meets the aerodynamic properties of drones.
\\\\

Many mobility models exist and is often based on the real wolrd.
This article specifically discusses two mobility models that are Random Mobility Model and distibuted Pheromon Model.\\\\

We tried to give the best possible image of our project through this memory. It represents each stage, from the study of the existing to the implementation of an algorithm. It also shows our organization as well as the multiple tests which we established and realized. 
