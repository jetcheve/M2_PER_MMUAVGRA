\chapter*{Introduction}

Dans le cadre de notre formation de Master Informatique à l'Université de Bordeaux 1 et notamment dans la matière : projet d'étude et de recherche, nous devions étudier un article de recherche par groupe de 5 personnes. Chaque groupe devait choisir un sujet de recherche parmi ceux proposés. Notre groupe a choisit un sujet en rapport avec les drones. Le titre de cet article est \textit{Mobility Models for UAV Group Reconnaissance Applications} écrit en collaboration par \textit{Erik Kuiper} et \textit{Simin Nadjm-Tehrani} publié en 2006. Les clients qui nous ont proposé d'étudier cet article sont messieurs Serge Chaumette et Vincent Autefage. Notre chargé de TD est monsieur Pascal Desbarats.
\\\\
Ce projet, d'une durée de 3 mois, nous a permis de voir le cheminement de l'étude d'un article de recherche, de sa lecture en anglais, à l'implémentation d'un algorithme, tout en passant par une étude de l'existant.
Le but de cette matière est donc de nous faire étudier un article de recherche, d'en extraire un algorithme et de l'implémenter. Nous devons donc étudier un modèle qui respecte les propriétés aérodynamiques des drones.\\
\\\\
As part of our Master Computer formation at the University of Bordeaux 1 particularly in the matter Study and Research Project, we had to study a research article in groups of 5 people. Every group had to choose a subject of research among those proposed. Our group chose a topic related to the drones. The title of this article is \textit{Mobility Models for UAV Group Reconnaissance Applications} written in collaboration by \textit{Erik Kuiper} and \textit{Simin Nadjm-Tehrani} published in 2006. The customers who suggested us studying this article are misters Serge Chaumette and Vincent Autefage. Our teaching assistant is mister Pascal Desbarats.
\\\\

This project, for a period of 3 months, we were able to see the progress of the study of a research article, its reading in English, with the implementation of an algorithm, while going through a study of the existing.
The aim of this material is to make us study a research article, to extract an algorithm from it and to implement it.
We must therefore study a model that meets the aerodynamic properties of drones.
\\\\
Plusieurs modèles de mobilité existent et se base le plus souvent sur des éléments du monde réel.
Cet article précisément étudie deux modèles de mobilité qui sont le Random Mobility Model et le Distibuted Pheromon Model.\\\\
Nous avons donc essayé de donner la meilleure image possible de notre projet à travers ce mémoire. Il représente chaque étapes de celui-ci, de l'étude de l'existant à l'implémentation d'un algorithme. Il montre également notre organisation ainsi que les multiples tests que nous avons établis et réalisés.
\\\\
This article specifically discusses two mobility models that are Random Mobility Model and distibuted Pheromon Model.\\\\
We tried to give the best possible image of our project through this memory. It represents each stage thereof, the study of the existing in the implementation of an algorithm. It also shows our organization as well as the multiple tests which we established and realized. 
