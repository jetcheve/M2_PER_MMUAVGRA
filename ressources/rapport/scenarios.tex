\part{Article}

\setcounter{chapter}{0}

\chapter{Scenarios}

We saw that the issue of this article is "how well scan an area?".\\
So the scenario is the reconnaissance of an area. It takes place in a square with a side length of 30 kilometers.\\
For this scenario, we use 10 UAVs and a command and control center (C\&C). Each UAV starts at the middle of the south edge heading north near the C\&C.
The UAVs have to scan entirely the map once per hour.\\

For the reconnaissance data, to be of any use, it needs to be transmitted to the users that need it. In this scenario, it's the C\&C which needs it. So each UAV need a communication path to C\&C to transmit the data collected. They have to maintain a contact with the C\&C, so they need to be close enough a neighbour UAV.\\
Fortunately a very simple model of communication is chosen for the scenario.  UAVs  within 8000  meters  of  each  other  can  communicate  with infinite  bandwidth.  If  they  are  further  away  no communication  is  possible.\\

Concerning the reconnaissance scan, an image resolution of 0.5 meters is chosen. The scan area is 2000x1000 meters thanks to an 8 megapixel camera and the image proportions 2:1 (width:lenght).